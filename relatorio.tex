%%%%%%%%%%%%%%%%%%%%%%%%%%%%%%%%%%%%%%%%%%%%%%%%%%%%%%%%%%%%%%%%%%%%%%%%%%%%%%%%%%%%%%
% Modelo de relatório de Disciplina de MLP a partir da
% classe latex iiufrgs disponivel em http://github.com/schnorr/iiufrgs
%%%%%%%%%%%%%%%%%%%%%%%%%%%%%%%%%%%%%%%%%%%%%%%%%%%%%%%%%%%%%%%%%%%%%%%%%%%%%%%%%%%%%%

% Ignore o comentario acima e imagine que exista um modelo de relatorio de CCI em algum lugar

%%%%%%%%%%%%%%%%%%%%%%%%%%%%%%%%%%%%%%%%%%%%%%%%%%%%%%%%%%%%%%%%%%%%%%%%%%%%%%%%%%%%%%
% Definição do tipo / classe de documento e estilo usado
%%%%%%%%%%%%%%%%%%%%%%%%%%%%%%%%%%%%%%%%%%%%%%%%%%%%%%%%%%%%%%%%%%%%%%%%%%%%%%%%%%%%%%
%
\documentclass{iiufrgs}

%%%%%%%%%%%%%%%%%%%%%%%%%%%%%%%%%%%%%%%%%%%%%%%%%%%%%%%%%%%%%%%%%%%%%%%%%%%%%%%%%%%%%%
% Importação de pacotes
%%%%%%%%%%%%%%%%%%%%%%%%%%%%%%%%%%%%%%%%%%%%%%%%%%%%%%%%%%%%%%%%%%%%%%%%%%%%%%%%%%%%%%
% (a A seguir podem ser importados os pacotes necessários para o documento, de acordo 
% com a necessidade)
%
\usepackage[brazilian]{babel}	    % para texto escrito em pt-br
\usepackage[utf8]{inputenc}         % pacote para acentuação
\usepackage{graphicx}         	    % pacote para importar figuras
\usepackage[T1]{fontenc}            % pacote para conj. de caracteres correto
\usepackage{times}                  % pacote para usar fonte Adobe Times
\usepackage{enumerate}              % para lista de itens com letras
\usepackage{breakcites}
%\usepackage{tikz}
%\usepackage[oldvoltagedirection]{circuitikzgit}
\usepackage{caption}
\usepackage{siunitx}
\usepackage{placeins}
\usepackage{titlesec}
\usepackage{enumitem}
\usepackage{titletoc}
\usepackage{subfig}
%\usepackage{listings}			    % para listagens de código-fonte
\usepackage{mathptmx}               % p/ usar fonte Adobe Times nas formulas matematicas
\usepackage{url}                    % para formatar URLs
%\usepackage{color}				    % para imagens e outras coisas coloridas
%\usepackage{fixltx2e}              % para subscript
%\usepackage{amsmath}               % para \epsilon e matemática
%\usepackage{amsfonts}
%\usepackage{setspace}			    % para mudar espaçamento dos parágrafos
%\usepackage[table,xcdraw]{xcolor}  % para tabelas coloridas
%\usepackage{longtable}             % para tabelas compridas (mais de uma página)
%\usepackage{float}
%\usepackage{booktabs}
%\usepackage{tabularx}
%\usepackage{hyperref}

\usepackage[alf,abnt-emphasize=bf]{abntex2cite}	% pacote para usar citações abnt

%%%%%%%%%%%%%%%%%%%%%%%%%%%%%%%%%%%%%%%%%%%%%%%%%%%%%%%%%%%%%%%%%%%%%%%%%%%%%%%%%%%%%%
% Macros, ajustes e definições
%%%%%%%%%%%%%%%%%%%%%%%%%%%%%%%%%%%%%%%%%%%%%%%%%%%%%%%%%%%%%%%%%%%%%%%%%%%%%%%%%%%%%%
%

% define estilo de parágrafo para citação longa direta:
\newenvironment{citacao}{
    %\singlespacing
    %\footnotesize
    \small
    \begin{list}{}{
        \setlength{\leftmargin}{4.0cm}
        \setstretch{1}
        \setlength{\topsep}{1.2cm}
        \setlength{\listparindent}{\parindent}
    }
    \item[]}{\end{list}
}

% adiciona a fonte em figuras e tabelas
\newcommand{\fonte}[1]{\\Fonte: {#1}}

\newcommand{\titlepagespecificinfo}{Relatório apresentado como requisito parcial para a obtenção de conceito na Disciplina de Métodos Ágeis.}
% \def\@cipspecificinfo{Concepção de Circuitos Integrados}


% Ative o seguinte caso alguma nota de rodapé fique muito longa e quebre entre múltiplas
% páginas
%\interfootnotelinepenalty=10000

%%%%%%%%%%%%%%%%%%%%%%%%%%%%%%%%%%%%%%%%%%%%%%%%%%%%%%%%%%%%%%%%%%%%%%%%%%%%%%%%%%%%%%
% Informações gerais                                   
%%%%%%%%%%%%%%%%%%%%%%%%%%%%%%%%%%%%%%%%%%%%%%%%%%%%%%%%%%%%%%%%%%%%%%%%%%%%%%%%%%%%%%

% título
\title{RELATÓRIO DO TRABALHO FINAL} 

% autor
%\author{Autores(s)}{Aluno(s)} % {sobrenome}{nome}
\author{Corrêa Pereira da Silva}{Henrique}

% Professor orientador da disciplina
\advisor[Profa.~Dra.]{Becker}{Karin}

% Nome do(s) curso(s):
\course{Curso de Graduação em Ciência da Computação}

% local da realização do trabalho 
\location{Porto Alegre}{RS} 

% data da entrega do trabalho (mês e ano)
\date{6}{2018}


% Palavras chave
\keyword{SCRUM}
\keyword{SCRUM Master}
\keyword{Retrospectiva}
\keyword{Métodos Ágeis}


%%%%%%%%%%%%%%%%%%%%%%%%%%%%%%%%%%%%%%%%%%%%%%%%%%%%%%%%%%%%%%%%%%%%%%%%%%%%%%%%%%%%%%
% Início do documento e elementos pré-textuais
%%%%%%%%%%%%%%%%%%%%%%%%%%%%%%%%%%%%%%%%%%%%%%%%%%%%%%%%%%%%%%%%%%%%%%%%%%%%%%%%%%%%%%

% Declara início do documento
\begin{document}

% inclui folha de rosto 
\maketitle

\selectlanguage{brazilian}

% Sumario
\tableofcontents



%%%%%%%%%%%%%%%%%%%%%%%%%%%%%%%%%%%%%%%%%%%%%%%%%%%%%%%%%%%%%%%%%%%%%%%%%%%%%%%%%%%%%
% Aqui comeca o texto propriamente dito
%%%%%%%%%%%%%%%%%%%%%%%%%%%%%%%%%%%%%%%%%%%%%%%%%%%%%%%%%%%%%%%%%%%%%%%%%%%%%%%%%%%%%

%espaçamento entre parágrafos
%\setlength{\parskip}{6 pt}

\selectlanguage{brazilian}

%%%%%%%%%%%%%%%%%%%%%%%%%%%%%%%%%%%%%%%%%%%%%%%%%%%%%%%%%%%%%%%%%%%%%%%%%%%%%%%%%%%%%
% Introdução
%

%Este capítulo tem o objetivo de descrever os detalhes necessários à correta formatação do documento. As informações aqui apresentadas devem ser suficientes para formatar corretamente o documento no ambiente \LaTeX.

%Os \textbf{Capítulos} são sempre iniciados com o comando \texttt{\char'134chapter}, que coloca-os em uma nova folha, em letras maiúsculas, numerados e  alinhados à esquerda. Para os \textbf{capítulos não-numerados} (Listas, Resumo, Abstract, Referências, etc.), o título é centralizado na linha Para tanto, usar o comando \texttt{\char'134chapter*}. Para ambos, são deixados 90 pt de espaçamento anterior (ou seja, distância da margem superior) e 42 pt de espaçamento posterior (espaço até o início do texto ou primeira subdivisão). 

%Todos os \textbf{demais parágrafos de texto} são escritos em espaçamento simples, com observância de 6 pt de espaçamento em relação ao parágrafo seguinte. O estilo atual já considera essas retrições. 


%As demais subdivisões do texto (seções, subseções, etc.) são formatadas com o título alinhado sempre à esquerda, precedido da respectiva numeração. Para tanto, no \LaTeX, você deve utilizar os comandos \texttt{\char'134section},  \texttt{\char'134subsection} e  \texttt{\char'134subsubsection}.

%São permitidas subdivisões até o 5º nível (onde o capítulo é o 1º. nível), porém no sumário inclui-se somente os títulos até o nível 3\footnote{O formato adotado pela ABNT prevê apenas três níveis (capítulo, seção e subseção).}. Assim, \texttt{\char'134subsubsection} não é aconselhado. 


%%%%%%%%%%%%%%%%%%%%%%%%%%%%%%%%%%%%%%%%%%%%%%%%%%%%%%%%%%%%%%%%%%%%%%%%%%%%%%%%%%%%%
% Intro
%

\chapter{Introdução}\label{intro}

Neste relatório, dissertarei sobre a importância de reuniões de retrospectiva no framework SCRUM e também sobre o papel do SCRUM \textit{master} nessas reuniões. 

No primeiro momento, focarei sobre os conceitos de melhoria incremental de métodos ágeis e sobre quais são as responsabilidades de um SCRUM \textit{master} na metodologia em si. Logo após, comentarei sobre as reuniões em si\footnote{no caso as \textit{Dailies}, as de \textit{Sprint} e as de Retrospectiva}, o que é abordado nelas, suas características e o comportamento esperado da equipe nelas. Por último, abordarei algumas atividades que podem ser realizadas numa reunião para obter resultados mais facilmente.

É importante lembrar que todas as observações realizadas neste relatório não correspondem a algum experiência prévia, sendo este um trabalho puramente teórico.


%%%%%%%%%%%%%%%%%%%%%%%%%%%%%%%%%%%%%%%%%%%%%%%%%%%%%%%%%%%%%%%%%%%%%%%%%%%%%%%%%%%%%
% Prefacio
%

\chapter{Prefácio}\label{pref}

Para o melhor entendimento de qualquer situação devemos possuir o correto contexto dessa mesma. Logo, neste capítulo abordarei alguns conceitos fundamentais de metodologias ágeis e questionarei o porquê por trás da decisão de adotar uma metodologia ágil. Não obstante, também abodarei o conceito de coaching, o papel do SCRUM \textit{master} no SCRUM e o que ele faz para alcançar seus objetivos.

Construído ou relembrado o contexto da discussão, estaremos prontos para discutir sobre as reuniões de fato.

\section{Sobre SCRUM}\label{scrum}



%%%%%%%%%%%%%%%%%%%%%%%%%%%%%%%%%%%%%%%%%%%%%%%%%%%%%%%%%%%%%%

\bibliographystyle{abntex2-alf}
\bibliography{biblio} 

\end{document}
