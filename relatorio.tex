%%%%%%%%%%%%%%%%%%%%%%%%%%%%%%%%%%%%%%%%%%%%%%%%%%%%%%%%%%%%%%%%%%%%%%%%%%%%%%%%%%%%%%
% Modelo de relatório de Disciplina de MLP a partir da
% classe latex iiufrgs disponivel em http://github.com/schnorr/iiufrgs
%%%%%%%%%%%%%%%%%%%%%%%%%%%%%%%%%%%%%%%%%%%%%%%%%%%%%%%%%%%%%%%%%%%%%%%%%%%%%%%%%%%%%%

% Ignore o comentario acima e imagine que exista um modelo de relatorio de CCI em algum lugar

%%%%%%%%%%%%%%%%%%%%%%%%%%%%%%%%%%%%%%%%%%%%%%%%%%%%%%%%%%%%%%%%%%%%%%%%%%%%%%%%%%%%%%
% Definição do tipo / classe de documento e estilo usado
%%%%%%%%%%%%%%%%%%%%%%%%%%%%%%%%%%%%%%%%%%%%%%%%%%%%%%%%%%%%%%%%%%%%%%%%%%%%%%%%%%%%%%
%
\documentclass{iiufrgs}

%%%%%%%%%%%%%%%%%%%%%%%%%%%%%%%%%%%%%%%%%%%%%%%%%%%%%%%%%%%%%%%%%%%%%%%%%%%%%%%%%%%%%%
% Importação de pacotes
%%%%%%%%%%%%%%%%%%%%%%%%%%%%%%%%%%%%%%%%%%%%%%%%%%%%%%%%%%%%%%%%%%%%%%%%%%%%%%%%%%%%%%
% (a A seguir podem ser importados os pacotes necessários para o documento, de acordo 
% com a necessidade)
%
\usepackage[brazilian]{babel}	    % para texto escrito em pt-br
\usepackage[utf8]{inputenc}         % pacote para acentuação
\usepackage{graphicx}         	    % pacote para importar figuras
\usepackage[T1]{fontenc}            % pacote para conj. de caracteres correto
\usepackage{times}                  % pacote para usar fonte Adobe Times
\usepackage{enumerate}              % para lista de itens com letras
\usepackage{breakcites}
\usepackage{caption}
\usepackage{siunitx}
\usepackage{placeins}
\usepackage{titlesec}
\usepackage{enumitem}
\usepackage{titletoc}
\usepackage{epigraph}
\usepackage{subfig}
\usepackage{csquotes}
%\usepackage{listings}			    % para listagens de código-fonte
\usepackage{mathptmx}               % p/ usar fonte Adobe Times nas formulas matematicas
\usepackage{url}                    % para formatar URLs
%\usepackage{color}				    % para imagens e outras coisas coloridas
%\usepackage{fixltx2e}              % para subscript
%\usepackage{amsmath}               % para \epsilon e matemática
%\usepackage{amsfonts}
%\usepackage{setspace}			    % para mudar espaçamento dos parágrafos
%\usepackage[table,xcdraw]{xcolor}  % para tabelas coloridas
%\usepackage{longtable}             % para tabelas compridas (mais de uma página)
%\usepackage{float}
%\usepackage{booktabs}
%\usepackage{tabularx}
%\usepackage{hyperref}

\usepackage[alf,abnt-emphasize=bf]{abntex2cite}	% pacote para usar citações abnt

%%%%%%%%%%%%%%%%%%%%%%%%%%%%%%%%%%%%%%%%%%%%%%%%%%%%%%%%%%%%%%%%%%%%%%%%%%%%%%%%%%%%%%
% Macros, ajustes e definições
%%%%%%%%%%%%%%%%%%%%%%%%%%%%%%%%%%%%%%%%%%%%%%%%%%%%%%%%%%%%%%%%%%%%%%%%%%%%%%%%%%%%%%
%

% define estilo de parágrafo para citação longa direta:
\newenvironment{citacao}{
    %\singlespacing
    %\footnotesize
    \small
    \begin{list}{}{
        \setlength{\leftmargin}{4.0cm}
        \setstretch{1}
        \setlength{\topsep}{1.2cm}
        \setlength{\listparindent}{\parindent}
    }
    \item[]}{\end{list}
}

% adiciona a fonte em figuras e tabelas
\newcommand{\fonte}[1]{\\Fonte: {#1}}

\newcommand{\titlepagespecificinfo}{Relatório apresentado como requisito parcial para a obtenção de conceito na disciplina de Métodos Ágeis.}

% Ative o seguinte caso alguma nota de rodapé fique muito longa e quebre entre múltiplas
% páginas
%\interfootnotelinepenalty=10000

%%%%%%%%%%%%%%%%%%%%%%%%%%%%%%%%%%%%%%%%%%%%%%%%%%%%%%%%%%%%%%%%%%%%%%%%%%%%%%%%%%%%%%
% Informações gerais                                   
%%%%%%%%%%%%%%%%%%%%%%%%%%%%%%%%%%%%%%%%%%%%%%%%%%%%%%%%%%%%%%%%%%%%%%%%%%%%%%%%%%%%%%

% título
\title{REUNIÕES RETROSPECTIVAS EM SCRUM NA PERSPECTIVA DO SCRUMMASTER} 

% autor
%\author{Autores(s)}{Aluno(s)} % {sobrenome}{nome}
\author{Corrêa Pereira da Silva}{Henrique}

% Professor orientador da disciplina
\advisor[Profa.~Dra.]{Becker}{Karin}

% Nome do(s) curso(s):
\course{Curso de Graduação em Ciência da Computação}

% local da realização do trabalho 
\location{Porto Alegre}{RS} 

% data da entrega do trabalho (mês e ano)
\date{6}{2018}


% Palavras chave
\keyword{Scrum}
\keyword{ScrumMaster}
\keyword{Retrospectiva}
\keyword{Métodos Ágeis}


%%%%%%%%%%%%%%%%%%%%%%%%%%%%%%%%%%%%%%%%%%%%%%%%%%%%%%%%%%%%%%%%%%%%%%%%%%%%%%%%%%%%%%
% Início do documento e elementos pré-textuais
%%%%%%%%%%%%%%%%%%%%%%%%%%%%%%%%%%%%%%%%%%%%%%%%%%%%%%%%%%%%%%%%%%%%%%%%%%%%%%%%%%%%%%

% Declara início do documento
\begin{document}

% inclui folha de rosto 
\maketitle

\selectlanguage{brazilian}

% Sumario
\tableofcontents

%%%%%%%%%%%%%%%%%%%%%%%%%%%%%%%%%%%%%%%%%%%%%%%%%%%%%%%%%%%%%%%%%%%%%%%%%%%%%%%%%%%%%
% Aqui comeca o texto propriamente dito
%%%%%%%%%%%%%%%%%%%%%%%%%%%%%%%%%%%%%%%%%%%%%%%%%%%%%%%%%%%%%%%%%%%%%%%%%%%%%%%%%%%%%

%espaçamento entre parágrafos
%\setlength{\parskip}{6 pt}

\selectlanguage{brazilian}

%%%%%%%%%%%%%%%%%%%%%%%%%%%%%%%%%%%%%%%%%%%%%%%%%%%%%%%%%%%%%%%%%%%%%%%%%%%%%%%%%%%%%
% Introdução
%

%Este capítulo tem o objetivo de descrever os detalhes necessários à correta formatação do documento. As informações aqui apresentadas devem ser suficientes para formatar corretamente o documento no ambiente \LaTeX.

%Os \textbf{Capítulos} são sempre iniciados com o comando \texttt{\char'134chapter}, que coloca-os em uma nova folha, em letras maiúsculas, numerados e  alinhados à esquerda. Para os \textbf{capítulos não-numerados} (Listas, Resumo, Abstract, Referências, etc.), o título é centralizado na linha Para tanto, usar o comando \texttt{\char'134chapter*}. Para ambos, são deixados 90 pt de espaçamento anterior (ou seja, distância da margem superior) e 42 pt de espaçamento posterior (espaço até o início do texto ou primeira subdivisão). 

%Todos os \textbf{demais parágrafos de texto} são escritos em espaçamento simples, com observância de 6 pt de espaçamento em relação ao parágrafo seguinte. O estilo atual já considera essas retrições. 


%As demais subdivisões do texto (seções, subseções, etc.) são formatadas com o título alinhado sempre à esquerda, precedido da respectiva numeração. Para tanto, no \LaTeX, você deve utilizar os comandos \texttt{\char'134section},  \texttt{\char'134subsection} e  \texttt{\char'134subsubsection}.

%São permitidas subdivisões até o 5º nível (onde o capítulo é o 1º. nível), porém no sumário inclui-se somente os títulos até o nível 3\footnote{O formato adotado pela ABNT prevê apenas três níveis (capítulo, seção e subseção).}. Assim, \texttt{\char'134subsubsection} não é aconselhado. 


%%%%%%%%%%%%%%%%%%%%%%%%%%%%%%%%%%%%%%%%%%%%%%%%%%%%%%%%%%%%%%%%%%%%%%%%%%%%%%%
% Intro
%

\chapter{Introdução}\label{intro}

Neste relatório, dissertarei sobre a importância de reuniões de retrospectiva no framework Scrum e também sobre o papel do \textit{ScrumMaster} nessas reuniões. 

No primeiro momento, focarei sobre os conceitos de melhoria incremental de métodos ágeis e sobre quais são as responsabilidades de um \textit{ScrumMaster} na metodologia em si. Logo após, comentarei sobre as reuniões em si\footnote{no caso as \textit{Dailies}, as de revisão da \textit{Sprint} e as de Retrospectiva da \textit{Sprint}}, o que é abordado nelas, suas características e o comportamento esperado da equipe nelas. Por último, abordarei algumas atividades que podem ser realizadas numa reunião para obter resultados mais facilmente.

É importante lembrar que todas as observações realizadas neste relatório não correspondem a alguma experiência prévia do autor deste relatório, sendo este um trabalho puramente teórico.


%%%%%%%%%%%%%%%%%%%%%%%%%%%%%%%%%%%%%%%%%%%%%%%%%%%%%%%%%%%%%%%%%%%%%%%%%%%%%%%
% Prefacio
%

\chapter{Prefácio}\label{pref}

Para o melhor entendimento de qualquer situação devemos possuir o correto contexto dessa mesma. Logo, neste capítulo abordarei alguns conceitos fundamentais de metodologias ágeis e questionarei o porquê por trás da decisão de adotar uma metodologia ágil. Não obstante, também abodarei o conceito de coaching, o papel do \textit{ScrumMaster} no Scrum e o que ele faz para alcançar seus objetivos.

Construído ou relembrado o contexto da discussão, estaremos prontos para discutir sobre as reuniões de fato.


% Sobre Scrum %%%%%%%%%%%%%%%%%%%%%%%%%%%%%%%%%%%%%%%%%%%%%%%%%%%%%%%%%%%%%%%%%

\section{Sobre Scrum}\label{Scrum}

Em grande parte das vezes, instituições tem grandes esperanças ao decidir adotar uma metodologia ágil. Dentre essas esperanças, podemos citar:

\begin{itemize}[leftmargin=3em, noitemsep, nosep, before=\vspace{1cm}, after=\vspace{1cm}]
    \setlength{\itemindent}{1em}
    \item Software de maior qualidade;
    \item Menor \textit{Time-To-Market};
    \item Menores custos de produção; 
    \item Maior satisfação do usuário final; 
    \item Mais features por entrega;
    \item Equipe mais engajada.
\end{itemize}

Além de outros motivos, claro. Tal é a esperança dessas instituições dentre muitas que gastam quantias consideráveis de dinheiro na transição para uma metodologia ágil, e que, assim, esperam resultados o mais cedo possível para justificar esses gastos.

O que \textit{CEO's}, diretores e gerentes esquecem é que ser ágil significa mais que dividir o desenvolvimento em ciclos de duas semanas. Ser ágil significa priorizar o cliente, dar poder ao time, e, acima de tudo, melhorar constantemente.

\subsection{Kaizen}\label{kaizen}

\epigraph{[...] there is something called standard work, but standards should be changed constantly. Instead, if you think of the standards as the best you can do, it's all over.}{-- \textit{Taiichi Ohno}}

Talvez alguém numa posição importante numa empresa tenha lido um livro sobre \textit{Extreme Programming} e tenha decidido que este é a metodologia correta pra empresa. Talvez essa pessoa tenha lido algum outro livro sobre alguma outra metodologia ágil e tenha chegado a conclusão que o \textit{approach} é perfeito para a situação da sua empresa. Muito provavelmente, essa pessoa está errada \cite{Cohn2009Succeeding}.

Nenhum processo é perfeito para a empresa em questão. Talvez algum deles seja um melhor ponto inicial que os outros, mas é inevitável que o processo deverá ser adaptado e customizado para a realidade desta empresa, o que será essencial para o sucesso dessa enquanto praticar alguma metodologia ágil. Por mais que muitos compartilhem a opinião do que significa Scrum neste exemplo, é impossível dizer como que será o estado \enquote{final} de um processo que busca melhoria contínua \cite{Cohn2009Succeeding}.

\subsection{\enquote{Melhores} práticas}\label{melhores}

Pelos motivos apresentados na subseção \ref{kaizen} que não podemos nos tentar a criar \enquote{mandamentos} ditando \enquote{melhores práticas}, qualquer que seja o escopo dessa \enquote{melhor} prática. Por mais tentadora que seja essa ideia, devemos evitar fazer isso, já que ao definir um limite superior\footnote{ou seja, \enquote{não podemos fazer melhor que isso} ou \enquote{esse é comprovadamente o melhor jeito de fazer isso}} estamos nos privando da melhoria constante, conceito central ao Scrum e à ideologia \textit{Kaizen}, conhecida pelo Modelo de Produção Toyota.

Logo, devemos sempre buscar e estimular nossas próprias práticas, que serão relevantes a nossa realidade no atual dado momento, sem seguir cegamente conselhos de gurus ou ditos \textit{\enquote{experts}}. 

\subsection{Conclusão}

Todos esses motivos apresentados culminam, finalmente, nos conceitos pregados pelo framework Scrum. Definido como um framework \textit{lightweight}, que, ao contrário de um processo como XP, não prescreve práticas técnicas e foca em práticas gerenciais. Tudo isso é alcançado ao analizar o andamento do projeto e do processo em si nas reuniões diárias, de revisão e de retrospectiva.


% Sobre Coaching %%%%%%%%%%%%%%%%%%%%%%%%%%%%%%%%%%%%%%%%%%%%%%%%%%%%%%%%%%%%%%

\section{Sobre coaching}\label{coaching}

Por definição, \textit{coaching} consiste de uma série de conversas hábeis em que o \textit{coach} ajuda o \textit{coachee} a ver novas perspectivas e possibilidades. A partir disso, o \textit{coachee} pode imaginar o seu próximo passo em direção ao seu crescimento profissional e pessoal e agir para colocar em ação esse passo \cite{Adkins2010Coaching}. No contexto de times ágeis, sim, teremos que dividir o papel do coach ágil entre \textit{coaching} de fato e entre mentorar.

Como coach ágil, você ainda está ajudando alguém a atingir seu próximo objetivo. Como coach ágil, você também está compartilhando experiências e valores, guiando o \textit{coachee} ao uso efetivo dos princípios Scrum, por exemplo. Com isso, nessa seção veremos o que significa ser e agir como um coach ágil e como utilizar esses conhecimentos para dar ao time a oportunidade de criar coisas incríveis.

\subsection{O que é}

\epigraph{Have compassion for each person's journey so they know you honor where they are as you help them become what they want to be. When they know--and feel--that you are loving and compassionate towards them, they can shed the posturing and preening and get real with you.}{-- \textit{Lyssa Adkins}}

%%%%%%%%%%%%%%%%%%%%%%%%%%%%%%%%%%%%%%%%%%%%%%%%%%%%%%%%%%%%%%

\bibliographystyle{abntex2-alf}
\bibliography{biblio} 

\end{document}
